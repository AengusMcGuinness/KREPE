% Created 2021-07-16 Fri 09:25
% Intended LaTeX compiler: pdflatex
\documentclass[11pt]{article}
\usepackage[utf8]{inputenc}
\usepackage[T1]{fontenc}
\usepackage{graphicx}
\usepackage{grffile}
\usepackage{longtable}
\usepackage{wrapfig}
\usepackage{rotating}
\usepackage[normalem]{ulem}
\usepackage{amsmath}
\usepackage{textcomp}
\usepackage{amssymb}
\usepackage{capt-of}
\usepackage{hyperref}
\author{Aengus McGuinness, and Erika Pedersen}
\date{\today}
\title{}
\hypersetup{
 pdfauthor={Aengus McGuinness, and Erika Pedersen},
 pdftitle={},
 pdfkeywords={},
 pdfsubject={},
 pdfcreator={Emacs 26.3 (Org mode 9.1.9)}, 
 pdflang={English}}
\begin{document}

\tableofcontents

\section{Compare-all.py: A Comprehensive Guide}
\label{sec:orgcc19e9d}
\subsection{Visualizations}
\label{sec:org969ff94}
Compare-all.py is a python command line tool within the krepe package
that allows bioniformaticists to count the kmers with any number of 
.fasta, .fna, or .fastq files, and also lets the user input any number
of files which are then parsed into kmers of length k and then visualise
d using a dendrogram. 
\subsection{Execution}
\label{sec:org19070f6}
After making the script executable using chmod +x you can run the program
with:
\$ ./\$PATH/compare-all.py (k-mer length desired) (file 1, 2, etc names with one space between them) (file type) (metadata or base)
Or you can use: 
\$ python3 /$PATH/compare-all.py (k-mer length desired) (file 1, 2 etc) (file type) (metadata or base)
\subsection{Tutorial}
\label{sec:orgd95578c}
Navigate to the tests folder and retrieve the three testing files. Move
them to the same directory as the python script. With your current 
directory set to the one with the program and test files, you can use the 
command format above to run the files through. Implementing the
command should look like this: 
\$ python3 compare-all.py 15 test1.fastq test2.fastq test3.fastq -fastq 

The compare-all.py has one option which is to include a label file which will allow you to directly label the dendrogram. This file should be formatted so that each line is a distinct label for the dendrogram.
\end{document}